\chapter{Conclusion}\label{chapter:conclusion}

In conclusion, in this application project we have developed new features for
the existing compute cloud infrastructure facilitating data access, storage, and
sharing of research data. The existing data visualizing service was made more
accessible to other affiliated researchers by implementing and configuring
DFN-AAI Shibboleth Login.

As a first step the existing prototype was separated into landing page and the
actual web application i.e. ParaViewWeb. This separation not only follows the
best practices in web development but also allows for easier maintenance and
future development by non-technical staff. Further, we have addressed a yet
unresolved issue with ParaView concerning incompatible libraries by downgrading
Ubuntu version to 20.04 LTS.

The primary contribution was integrating Shibboleth into the existing
infrastructure. This involved configuring Shibboleth SP to handle authentication
and authorization, and access control with Apache to ensure secure and seamless
access to protected resources. The SP was configured to work with the DFN-AAI
test federation, with plans to transition to the productive version upon
successful testing and correct DNS records set. Key steps included installing
necessary software components, generating X.509 certificates for SAML message
signing, and configuring Apache to work with Shibboleth.

Finally, a new user portal was developed in order to automize new user creation
for our service. This portal consists of a form collecting user information,
including LRZ username and DSS UID, which are required when a new user wants to
share data from a DSS container. The form data is securely transmitted to the
server using the POST method, invoking a server-side script creating a new user,
while protecting the portal with a required DFN-AAI login.

The next steps include setting DNS records for the subdomains, registering our
service's metadata with the DFN-AAI federation and changing Shibboleth SP to use
the productive federation. Future work could focus on further optimizing the
compute cloud infrastructure, exploring additional security measures, and
expanding the scope not only to data visualizing but supporting other
applications including custom code. By continuing to build on the achievements
of this project, we can ensure that valuable research data remains accessible
and useful to the scientific community.
