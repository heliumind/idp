\chapter{Implementation}\label{chapter:implementation}

This chapter contains detailed information on the rationale and how solutions
are implemented. It also gives instructions on how to install them. It assumes
the reader has ssh access to the server, sudo privileges, access to the source
code and has cloned the code repository on the server as detailed in appendix
\ref{chapter:code}.

\section{Existing Data Visualization Service}

The existing service, as detailed in~\cite{nowak2024pvw}, provides a web-based
framework for sharing and visualizing research data. The visualization
functionality leverages Kitware's ParaViewWeb Framework for browser-based remote
rendering. The ParaView-like web application allows users to directly visualize
simulation results interactively in their web browser without the need to
download large datasets. The service is realized on a LRZ Compute Cloud (CC)
instance serving ParaView Visualizer and an Apache HTTP server functioning as a
reverse proxy.

It seamlessly integrates with LRZ's Data Science Storage (DSS) as it is
connected to the CC instance via an NFS mount. Users can share specific folders
from their personal DSS directory with external researchers who do not have
direct access to the LRZ. This is achieved through unique access links, similar
to file-sharing services like Google Drive or Dropbox.

\subsection{Fixing ParaView OSMesa and glibc Issue}

In February Ubuntu 22.04 LTS upgraded its GNU C Library, commonly known as
\textit{glibc}, to version 2.35. This broke the existing data visualization
service as the used ParaView Server is incompatible with the upgraded
\textit{glibc}. The underlying problem is the \textit{OSMesa} binary used by
ParaView for software rendering and is further described in this GitLab
Issue~\cite{kitwareOsmesaBinary}. Until this report's submission date there has
not been a fix for thi issue.

As proposed by the ParaView maintainers in~\cite{paraviewUsingUbuntu}
workarounds for this bug are

\begin{enumerate}
    \item Using \textit{EGL} flavor of ParaView Server
    \item Compiling ParaView with \textit{OSMesa} locally
    \item Downgrading \textit{glibc}
\end{enumerate}

Option 1 requires a CC instance with a graphics card as the \textit{EGL} variant
uses hardware rendering. However as currently the demand for GPUs is
unquestionably high, they are no instances with GPU available. Option 2 is an
involved task with software projects of considerable size as in the case of
ParaView, making it also infeasible. Thus, we have decided on option 3 by
setting up a new CC instance running on Ubuntu 20.04 LTS that comes with
\textit{glibc} version 2.31 and installing the data visualization service again.
This is because \textit{glibc} is a very common dependency and downgrading it in
a live system would lead to further incompatibilities with other software.

\subsubsection*{Instructions for the workaround}

For the most part appendix B (Anleitung zur Bereitstellung)
of~\cite{nowak2024pvw} can be followed to install the existing service and
helper scripts on a new CC instance. Note that as the source code is not
publicly available anymore instead of using curl to download needed files, we
assume the reader already has every file needed by cloning the code repository.
After following appendix~\ref{chapter:code} to do so, the reader would need to
change to the directory that contains needed files. 

For example instead of
\begin{terminal}
    curl -fSLOJ https://raw.githubusercontent.com/FabianNowak/pv-visualizer-multi-project-config/releases/<directory>/<needed-file>
\end{terminal}

The following suffices
\begin{terminal}
    cd <directory>
\end{terminal}

The custom executable \texttt{pv-session-mapper} written in Rust has to be
compiled locally as \textit{glibc} is dynamically linked. This means at section
\textbf{pv-session-mapper installieren} follow these instruction instead

\begin{enumerate}
    \item First install Rust and its build system Cargo by using \textit{rustup}
    \begin{terminal}
        curl https://sh.rustup.rs -sSf | sh
    \end{terminal}
    It will download a script, and start the installation. If everything went
    well \texttt{Rust is installed now. Great!} should appear.
    \item Change into source directory and compile locally
    \begin{terminal}
        cd pv-session-mapper/session_mapper
        cargo build --release
    \end{terminal}
    \item Create target directory, copy and adjust custom binary
    \begin{terminal}
        sudo mkdir /opt/pv-session-mapper
        sudo cp target/release/pv-session-mapper /opt/pv-session-mapper
        sudo chown root:root /opt/pv-session-mapper/pv-session-mapper
        sudo chmod +x /opt/pv-session-mapper/pv-session-mapper
    \end{terminal}
\end{enumerate}

Once through the tutorial, these final adjustments on the Apache Server have to
be made

\begin{enumerate}
    \item Activate Apache Server module \texttt{proxy\_wstunnel} with 
    \begin{terminal}
        sudo a2enmod proxy_wstunnel
    \end{terminal}
    \item Restart Apache Server to load new configuration with
    \begin{terminal}
        sudo systemctl restart apache2
    \end{terminal}
\end{enumerate}

\subsection{Reorganizing Apache HTTP Server configuration}

The existing data visualization service will need a publicly accessible landing
page and a protected new user portal after SSO login via DFN-AAI was successful.
It is best practice to separate the web app i.e. ParaViewWeb from the marketing
site. Benefits are that non-technical staff can independently modify content of
the landing page without possibly breaking the web app. One way to achieve a
separation is with the help of subdomains. 

In our design, the landing page will be reachable under
\texttt{marge.aer.ed.tum.de} whereas the service will be bound to the subdomain
\texttt{pvw.marge.aer.ed.tum.de}. This setup also simplifies DNS configuration,
the system responsible for translating human-readable names to actual routable
IP-addresses, as only two records are needed. An A-Record for linking
\texttt{marge.aer.ed.tum.de} to the CC instance's floating IP-address as well as
a wildcard DNS record linking \texttt{*.marge.aer.ed.tum.de} to the same
IP-address. The wildcard record will match for every subdomain, even not yet
existing ones, making this setup suitable for future requirements. 

In order to achieve aforementioned behavior, Apache's virtual host feature is
leveraged, allowing different content to be served over the same Apache server,
depending on the host name used in the request. ParaViewWeb continues to use its
existing subdirectory whereas the landing page is served from a new
subdirectory.

This means the existing \texttt{paraview-multi-user.conf} is to be moved from
\texttt{conf-available} to \texttt{sites-available} directory, the usual place
where virtual configurations are placed, slightly changed to account for the
subdomain as seen in Listing~\ref{lst:002-pvw} and a new landing page
configuration shown in Listing~\ref{lst:001-marge} is to be added. Note how
Apache resolves which virtual host to serve with the help of the configured
\texttt{ServerName}.

The new landing page configuration resembles a standard site configuration and
is constructed in the following way:
\begin{itemize}
    \item Virtual host configurations are enclosed in a \texttt{<VirtualHost>}
    directive, meaning that everything nested within only applies to the defined
    VirtualHost.
    \item In line 2, \texttt{ServerName} specifies under which name the
    landing page is accessible.
    \item \texttt {DocumentRoot} in line 5 specifies which directory serves
    the landing page's content.
    \item \texttt{ErrorLog} and \texttt{CustomLog} define in which
    location the virtual host's error and access logs are saved.
    \texttt{APACHE\_LOG\_DIR} is a provided variable and resolves to the
    standard location \texttt{/var/log/apache2}.
    \item Lines 10-14 allow the Apache server to access the files provided in
    line 5.
\end{itemize}

\lstinputlisting[
    style=sourcecode,
    float=htp,
    label={lst:001-marge},
    caption={\texttt{001-marge.conf}: Apache configuration for Landing Page}
]{../apache-config/sites-available/001-marge.conf}

The changes made to ParaViewWeb's Apache configuration are minor and consist of
enclosing the configuration in a \texttt{<VirtualHost>} directive,  in line 2
and specifying an own subdirectory \texttt{/var/log/apache2/pvw} for storing the
web app's Apache logs in line 12-13 to separate concerns.

\lstinputlisting[
    style=sourcecode,
    float=htp,
    label={lst:002-pvw},
    caption={\texttt{002-pvw.conf}: Apache configuration for ParaViewWeb}
]{../apache-config/sites-available/002-pvw.conf}

\subsubsection*{Instructions}

\begin{enumerate}
    \item Create a folder to serve files for landing page
    \texttt{marge.aer.ed.tum.de}
    \begin{terminal}
        sudo mkdir /var/www/html/marge
    \end{terminal}
    \item Create a folder to store ParaViewWeb's log files
    \begin{terminal}
        sudo mkdir /var/log/apache2/pvw
    \end{terminal}
    \item Adjust \texttt{pv-configurator} settings to reflect ParaViewWeb
    subdomain. Open and edit \texttt{configurator\_settings.json} with root
    privileges
    \begin{terminal}
        sudo vim /srv/pv-configurator/configurator_settings.json
    \end{terminal}
    \item Make following change and save file
    \begin{lstlisting}[basicstyle=\ttfamily\small, frame=single]
        {
            "servername": "pvw.marge.aer.ed.tum.de",
            ...
        }
    \end{lstlisting}
    \item Change into configuration directory and copy both into apache
    configuration
    \begin{terminal}
        cd apache-config/sites-available
        sudo cp 001-marge.conf /etc/apache2/sites-available
        sudo cp 002-pvw.conf /etc/apache2/sites-available
    \end{terminal}
    \item Deactivate old ParaViewWeb Apache configuration
    \begin{terminal}
        sudo a2disconf paraview-multi-project
    \end{terminal}
    At this point \texttt{paraview-multi-project.conf} could be deleted as it
    is no longer used.
    \item Finally activate new configuration and restart Apache server
    \begin{terminal}
        sudo a2ensite 001-marge
        sudo a2ensite 002-pvw
        sudo systemctl restart apache2
    \end{terminal}
\end{enumerate}

\section{Creating a New User Portal}

The existing data visualization service allows users without LRZ credentials to
access shared projects via a unique access link. We want to extend the sharing
functionality to the same target group of non LRZ users. Additionally, new user
creation currently requires human involvement in processing such a request in
authenticating and creating a new Linux user on the CC instance. The goal is to
automize user management with the help of a self-service new user portal that
any authenticated user by the DFN-AAI can access. 

Our solution proposition consists of a form which is protected by DFN-AAI login,
collects necessary data and upon submission runs a server side script in the
background to create a new Linux user. Data collected include the new user's
first name, last name from which a username is generated as well as a desired
password for the new linux account. Further depending on whether the new user
has data on DSS to share, lrz username (which will be the linux username instead
of the generated) and \textit{userid} on DSS are collected. The script shall add
the user to necessary groups to allow ParaView sharing usage and set correct
\textit{userid} for DSS folder access. DSS connection has to still be done by
hand as detailed in appendix D (Anleitung: DSS-Verzeichnis mittels bindfs mit
anderen Nutzern auf der Compute-Cloud-Instanz teilen) of~\cite{nowak2024pvw}.

We realized the form with the help of a HTML form, which upon submission runs a
PHP script validating input and finally executes with sudo privileges a bash
script creating the actual linux user. PHP has been chosen as it is well
integrated with Apache, making it a common choice for web-based applications on
Linux servers. Security is of concern as this setup allows users to execute
privileged commands on the underlying linux server. This risk will be mitigated
by thoroughly sanitizing and validating user input in the server-side PHP
script. In addition proper access control and logging are set in place, as
further described in section~\ref{section:logging}.

\subsection{HTML form}

\lstinputlisting[
    language=html,
    style=sourcecode,
    linerange={1-2,24-48},
    float=htp,
    label={lst:htmlform},
    caption={\texttt{index.html}: New User Form}
]{../newuser/index.html}

Listing~\ref{lst:htmlform} depicts the form collecting user information needed
for linux account creation and is built the following way:

\begin{itemize}
    \item The \texttt{<form>} definition in line 5 ensures that from data will
    be sent using a \textit{POST} method and points to \texttt{create\_user.php}
    script on the server handling user creation.
    \item Line 6-13 define basic form fields for first name, last name, email
    and password. Each consists of a \texttt{label} and \texttt{input} for text,
    email and password fields respectively. The \texttt{for} attribute in the
    label is associated with the \texttt{id} of the input field to ensure
    accessibility. The \textit{name} attribute in the input field identifies
    this field with its value in the form data sent to the server. Email and
    password input types come with built-in email validation and hides entered
    characters are hidden from view.
    \item Line 15-16 is a checkbox input for users to indicate whether they want
    to share data on DSS and used to conditionally show the second set of
    fields and is further explained below.
    \item Second set of fields from line 17-22 wrapped in a \texttt{<fieldset>}
    tag are for inputting \textit{LRZ username} and \textit{uid on DSS} and also
    consist of an \texttt{label} and \texttt{input} tag. The \textit{uid} input
    is restricted to numbers between 1000 and 59999, the range uids are
    typically reserved for users in Linux distributions.
    \item Finally line 24 is a submit button that sends the form data to the
    server when clicked.
\end{itemize}

\lstinputlisting[
    language=html,
    style=sourcecode,
    linerange={3-23},
    float=ht,
    label={lst:formstyle},
    caption={\texttt{index.html}: New User Portal Stylesheet}
]{../newuser/index.html}

The \texttt{<style>} block in \texttt{index.html} seen in
Listing~\ref{lst:formstyle} contains CSS rules for showing and hiding the input
fields for \textit{LRZ username} and \textit{uid on DSS} conditionally based on
the state of the checkbox. The rules are defined for classes \textit{control}
and \textit{conditional} and are applied respectively by assigning UI elements
with the corresponding class and works as follows:

\begin{itemize}
    \item First block specifies when class \texttt{.control} is checked then the
    \texttt{.conditional} elements that follow them in the DOM are styled to be
    visible with properties from line 3-8. 
    \item Second block specifies when class \textit{.control} is not checked
    then the \textit{.conditional} elements are styled to be hidden with
    properties from line 12-19. These properties effectively hide the
    \texttt{.conditional} elements from view while maintaining a minimal
    footprint in the
    layout.
\end{itemize}

\subsection{Create User PHP Script}

\lstinputlisting[
    language=php,
    style=sourcecode,
    linerange={1-1, 9-43},
    float=htp,
    label={lst:phpvalidate},
    caption={\texttt{create\_user.php}: Server-Side Script Validating User Input}
]{../newuser/create_user.php}

\lstinputlisting[
    language=php,
    style=sourcecode,
    linerange={45-55, 58-59, 61-64},
    float=ht,
    label={lst:phpexecute},
    caption={\texttt{create\_user.php}: Server-Side Script Executing Shell Script}
]{../newuser/create_user.php}

\subsection{Bootstrap User Shell Script}

% \lstinputlisting[
%     language=bash,
%     style=sourcecode,
%     linerange={1-2, 45-79},
%     float=htp,
%     label={lst:bootstrap},
%     caption={\texttt{bootstrap\_user.sh}: Create New User Shell Script}
% ]{../newuser/bootstrap_user.sh}

\subsection{Adding Logging}\label{section:logging}

\subsubsection*{Instructions}

\begin{enumerate}
    \item Install php 
    \begin{terminal}
        sudo apt install php libapache2-mod-php
    \end{terminal}
    \item Create target directory to serve New User Portal
    \begin{terminal}
        sudo mkdir /var/www/html/marge/newuser
    \end{terminal}
    \item Create log file and modify file permissions
    \begin{terminal}
        sudo mkdir /var/log/newuser
        sudo touch /var/log/newuser/create_user_php.log
        sudo chown apache-proxy:apache-proxy /var/log/newuser/create_user_php.log
        sudo chmod 640 /var/log/newuser/create_user_php.log
    \end{terminal}
    \item Change into the repository's newuser directory and copy HTML form to
    collect the new user's information, server-side script into target directory
    \begin{terminal}
        cd newuser
        sudo cp index.html /var/www/html/marge/newuser
        sudo cp create_user.php /var/www/html/marge/newuser
        sudo cp bootstrap_user.sh /var/www/html/marge/newuser
    \end{terminal}
    \item Modify file ownerships and make bootstrap user shell script executable
    \begin{terminal}
        sudo chown root:root /var/www/html/marge/newuser/*
        sudo chmod +x /var/www/html/marge/newuser/bootstrap_user.sh
    \end{terminal}
    \item Create new entry to sudoers configuration by editing the file
    \begin{terminal}
        sudo visudo /etc/sudoers.d/bootstrap_user
    \end{terminal}
    \textbf{Only ever make changes with visudo, otherwise sudo can no longer be
    used in case of a syntax error.} Visudo warns if an attempt is made to save
    an incorrect file.
    \item Insert following single line and save the configuration file
    \begin{lstlisting}[frame={l}]
        apache-proxy ALL=(ALL) NOPASSWD: /var/www/html/marge/newuser/bootstrap_user.sh
    \end{lstlisting}
    This allows Apache's runtime user \texttt{apache-proxy} to execute the
    bootstrap user shell with sudo privileges.
    \item Finally restart Apache server
    \begin{terminal}
        sudo systemctl restart apache2
    \end{terminal}
\end{enumerate}

At this point the New User Portal can be accessed at
\texttt{marge.aer.ed.tum.de/newuser}.

\section{Installing and Configuring Shibboleth SP}\label{section:shibboleth}

\subsection{Password Protecting New User Portal}
