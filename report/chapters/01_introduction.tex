% !TeX root = ../main.tex
% Add the above to each chapter to make compiling the PDF easier in some editors.

\chapter{Introduction}\label{chapter:introduction}

\section{Motivation}

An essential part of science is the collaboration between different research
groups and the associated exchange of research results and research data.
Re-analysis of data by other research teams helps to verify the results. This is
a crucial component of the research process. In the same way, alternative
approaches to interpretation can contribute to scientific
progress~\cite{tenopir2011data}. It has also been found that publications with
openly accessible data are subsequently cited more often (up to 30
\%)~\cite{piwowar2013data}. In addition, the use of existing data can save the
resources required for new data collection. 

The Chair of Aerodynamics and Fluid Mechanics at TUM is involved in many
different areas of research, including flow simulation. Flow simulations are an
alternative to experiments wind tunnel and can be used in situations where an
experimental test setup is difficult. They allow various physical phenomena and
chemical effects that occur, for example, in flows at hypersonic speeds, to be
analyzed more precisely. Especially for computational fluid dynamics (CFD)
simulations, a lot of computing time is spent on high-performance computing
clusters. It is therefore important to be able to disseminate the results as
easily as possible.

\section{National Research Data Infrastructure for Engineering Sciences
(NFDI4Ing)}

This calls for research processes as well as the resulting solutions to be
accompanied by a proper research data management (RDM) that implements the FAIR
data principles: data has to be \textit{findable}, \textit{accessible},
\textit{interoperable} and \textit{re-usable}. The National Research Data
Infrastructure for Engineering Sciences (NFDI4Ing)~\cite{schmitt2020nfdi4ing}
brings together the engineering communities to work towards that goal. As part
of the German National Research Data Infrastructure (NFDI), the consortium aims
to develop, disseminate, standardize and provide methods and services to make
engineering research data FAIR. 

In order to cater to the different needs in research data management in the
engineering sciences, NFDI4Ing is divided into several ideal-typical
representatives, the so-called archetypes. The Chair of Aerodynamics and Fluid
Mechanics coordinates the \textit{DORIS} archetype: research on high-performance
computers (HPC) with very large amounts of data. 

\section{Aim of this work}
In this context, the proposed application project aims to investigate the
process of making large amounts of research data, obtained from a simulation of
the re-entry flow of an Apollo-like space capsule, available to a restricted
audience without Leibniz Supercomputing Centre (LRZ) credentials. The study will
focus on developing new features for an existing compute cloud infrastructure to
facilitate data access, storage, and sharing. The primary objective is to extend
the existing data visualizing service from~\cite{nowak2024pvw} to enable
seamless data availability and collaboration with other affiliated researchers
by enabling and configuring the DFN-AAI Shibboleth Login by the German National
Research and Education Network (DFN). \\

% TODI: intro about how the rest is aufgebaut
